\documentclass[a4paper,12pt]{article}

\usepackage{amsthm}
\usepackage{amssymb}
\usepackage{amsmath}
\usepackage{hyperref}
\usepackage{xifthen}
\usepackage{xparse}
\usepackage{dsfont}
\usepackage{xcolor}

\usepackage{fullpage}

% Left-right bracket
\newcommand{\lr}[1]{\left (#1\right)}

% Left-right square bracket
\newcommand{\lrs}[1]{\left [#1 \right]}

% Left-right curly bracket
\newcommand{\lrc}[1]{\left \{#1\right\}}

% Left-right absolute value
\newcommand{\lra}[1]{\left |#1\right|}

% Left-right upper value
\newcommand{\lru}[1]{\left \lceil#1\right\rceil}

% Scalar product
\newcommand{\vp}[2]{\left \langle#1, #2 \right \rangle}

% The real numbers
\newcommand{\R}{\mathbb{R}}

% The natural numbers
\newcommand{\N}{\mathbb{N}}

% A nicer emptyset symbol
\let\emptyset\varnothing

% Set constructor
\newcommand{\setdef}[2]{\lrc{ #1 \; \text{\textbar} \; #2}}

% j'th entry
\newcommand{\xj}{X^{(j)}}

% pretty epsilon
\renewcommand{\epsilon}{\varepsilon}

% eps, delta - dp
\newcommand{\edp}{$(\epsilon, \delta)$-DP }

% rando algo M
\newcommand{\M}{\mathcal{M}}

% Probability
\newcommand{\prob}[1]{\Pr\lrs{#1}}

% Expected Error
\newcommand{\ee}[1]{\mathbb{E} \lrs{ \| #1 \|^2}}

% Variance 
\newcommand{\var}[1]{\text{Var} \lrs{ #1 }}

% Expected value
\newcommand{\Exp}[1]{\mathbb{E} \lrs{ #1 }}

% Mean estimate
\newcommand{\mue}{\bm{\hat{\mu}}}

% (epsilon, delta)
\newcommand{\ed}{\lr{\epsilon, \delta}}

% Cite attempt
\newcommand{\defcite}[2]{\cite[#2]{#1}}

\usepackage[utf8]{inputenc}
\usepackage{graphicx}  % For including images

\usepackage{amsfonts}
\usepackage{todonotes}
\usepackage{hyperref}
\usepackage{cleveref}  % For referencing equations
\usepackage{cite} % For bibtex

% Used for writing pseudo code
\usepackage{algorithm}
\usepackage{algpseudocode}
% defined for writing input/output in pseudo code
\algblock{Input}{EndInput}
\algnotext{EndInput}
\algblock{Output}{EndOutput}
\algnotext{EndOutput}
\newcommand{\Desc}[2]{\State \makebox[6em][l]{#1}#2}

\usepackage[smartEllipses]{markdown}

\renewcommand{\qed}{\hfill\blacksquare}
\renewenvironment{proof}{{\bfseries Proof \\}}{\qed}

\newtheorem{definition}{Definition}[section]
\newtheorem{theorem}{Theorem}


\title{Invetigation of the Elliptical Gaussian Noise in the case of multivariate normal data}
\author{Tim Sehested Poulsen}

\begin{document}

\maketitle

\section{Preliminaries}
\subsection{Definitions}

\textbf{The dataset} is denoted by $X$, where 
$X \in \R^{n \times d}$.
I have that $n$ denotes the number of entries in the dataset and 
$d$ is the number of dimensions of the dataset.
I will throughout the report refer to a single entry of 
the dataset as $x_i$ and a single dimension of the dataset as $\xj$, 
and therefore $x^{(j)}_i$
denotes the $j$'th dimension of the $i$'th entry.
\vspace*{0.3cm}

\textbf{Differential privacy} is the heuristic of 
releasing a database statistic whilst limiting the impact
of any one entry. It builds on the intuition that computing
a statistics on a private dataset should not reveal 
any sensitive information about any one individual 
as long as that individual has little to no effect on the outcome.
Differential privacy has multiple slightly different
formal definitions, 
one such is $(\epsilon, \delta)$-Differential Privacy
refered to as \edp which will be introduced later on.
A prerequisite for almost all of the different differential privacy
definitions relies on the concept of neighbouring dataset.
\vspace*{0.3cm}

\begin{definition}[Neighbouring dataset \cite{dwork2016}]
Two dataset $X, X' \in \R^{n \times d}$ are said to be 
neigbouring if they differ in at most a single entry.
Neighbouring dataset are denoted with the relation $X \sim X'$ and defined as followed
\[ X \sim X' \iff \lra{ \setdef{i \in \N}{i \le n \land x_i \ne x_i' } } \le 1 \]
\end{definition}

\begin{definition}[Sensitivity \cite{Lebeda2022}]
Let $f(X): \R^{n \times d} \rightarrow \R^d$ given by 
$f(X) = \sum_{i = 1}^n x_i$ be the sum over all vectors in a dataset.
The sensitivity is then the maximal 
possible difference in the output of our summation  
from two neighbouring dataset denoted as $\Delta$.
We denote the sensitivity of the $j$'th dimension as
\[ \Delta_j = \max_{X \sim X'} \lra{{f(X)}^{(j)} - {f(X')}^{(j)}} \]
and then the total $l_2$-sensitivity is then
\[ \| \Delta \| = \max_{X \sim X'} \left\| {f(X)} - {f(X')} \right\|  \]
    
\end{definition}

\vspace*{0.3cm}
\begin{definition}[$(\epsilon, \delta)$-Differential Privacy \cite{dwork2016}]
A randomized algorithm $\M: \R^{n \times d} \rightarrow \mathcal{R}$ 
is $(\epsilon, \delta)$-differentially private if for all possible 
subsets of outputs $S \subseteq \mathcal{R}$ and all pairs of 
neighbouring dataset $X \sim X'$ we have that
\[ \prob{M(X) \in S} \le e^{\epsilon} \cdot \prob{M(X') \in S} + \delta \]

    
\end{definition}


\section{Algorithms}
\subsection{The Gaussian Mechanism}
One of the most foundational algorithms for achieving 
\edp is the Gaussian Mechanism \cite{dpbasic}. It computes the real
value of a statistic, where the $l_2$-sensitivity is known.
That is it produces a \edp estimate of a function 
$g: \R^{n \times d} \rightarrow \R^{d}$ where $ \| \Delta \|$ 
for the function $g$ is known.
It does so by computing the value of $g(X)$ and then adding noise
to each dimension from the normal distribution 
$\mathcal{N}(0, \sigma_{\epsilon,\delta}^2)$.
This can be seen as adding a noise vector $\eta$ which 
is then distributed according to
the multivariate normal distribution 
$\mathcal{N}(\overrightarrow{0}, \sigma_{\epsilon,\delta}^2I)$. 
The algorithm can be seen in Algorithm \ref{alg:gaussmech}.

\begin{algorithm}
\caption{The Gaussian Mechanism}\label{alg:gaussmech}
\begin{algorithmic}
    \Input
    \Desc{$\sigma_{\epsilon,\delta}$}{Standard deviation required to achieve \edp}
    \Desc{$X \in \R^{n \times d}$}{Dataset}
    \EndInput
    \Output
    \State\edp estimate of $g(X)$
    \EndOutput
    \State$\eta \gets \textnormal{sample from } \mathcal{N}(\overrightarrow{0}, \sigma_{\epsilon,\delta}^2I)$ \\
    \Return$g(X) + \eta$
\end{algorithmic}
\end{algorithm}
It is quite apparent that the main difficulty of the mechanism 
lies in determining a $\sigma_{\epsilon, \delta}$ which achieves
\edp, and preferably the smallest such one.

The following theorem was initially proven
\begin{theorem}\textnormal{\cite{dpbasic}}
Let $g: \R^{n \times d} \rightarrow \R^{d}$ be an arbitrary 
$d$-dimensional function with $l_2$-sensitvity 
$\| \Delta \| = \max_{X \sim X'} \| g(X) - g(X') \|$, 
and let $\epsilon \in (0,1)$.
The Gaussian Mechanism with
$\sigma_{\epsilon, \delta} = \| \Delta \| \sqrt{2\ln(1.25/\delta)}/\epsilon$ 
is \edp.
\end{theorem}
The proof is rather long and is therefore ommitted here.

-Does introduce the same variance for all dimensions regardless of individual variance

\section{Problem setup}
The problem consists of realeasing the sum of vectors 
in a dataset under differential privacy.
More formally we whish to release the value of 
$f:\R^{n \times d} \rightarrow  \R^d$ given by
\[ f(X) = \sum_{i = 1}^n x_i  \] 
under \edp. \\
The problem that the Elliptical Gaussian Mechanism solves 
is in the setting where all dimensions $\xj$ are restricted
by some bound $\Delta_j$ \cite{Lebeda2022}. This means that all 
$ x^{(j)}_i \in \lrs{-\Delta_j/2, \Delta_j/2} $.


\bibliography{refs.bib}{}
\bibliographystyle{acm}


\end{document}

