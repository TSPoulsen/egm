\documentclass[a4paper,12pt]{article}

\usepackage{amsthm}
\usepackage{amssymb}
\usepackage{amsmath}
\usepackage{hyperref}
\usepackage{xifthen}
\usepackage{xparse}
\usepackage{dsfont}
\usepackage{xcolor}

\usepackage{fullpage}

% Left-right bracket
\newcommand{\lr}[1]{\left (#1\right)}

% Left-right square bracket
\newcommand{\lrs}[1]{\left [#1 \right]}

% Left-right curly bracket
\newcommand{\lrc}[1]{\left \{#1\right\}}

% Left-right absolute value
\newcommand{\lra}[1]{\left |#1\right|}

% Left-right upper value
\newcommand{\lru}[1]{\left \lceil#1\right\rceil}

% Scalar product
\newcommand{\vp}[2]{\left \langle#1, #2 \right \rangle}

% The real numbers
\newcommand{\R}{\mathbb{R}}

% The natural numbers
\newcommand{\N}{\mathbb{N}}

% A nicer emptyset symbol
\let\emptyset\varnothing

% Set constructor
\newcommand{\setdef}[2]{\lrc{ #1 \; \text{\textbar} \; #2}}

% j'th entry
\newcommand{\xj}{X^{(j)}}

% eps, delta - dp
\newcommand{\edp}{$(\epsilon, \delta)$-DP}

% rando algo M
\newcommand{\M}{\mathcal{M}}

% Probability
\newcommand{\prob}[1]{\textnormal{Pr}\lrs{#1}}



\usepackage[utf8]{inputenc}
\usepackage{graphicx}  % For including images
\usepackage{bm}  % For bold math symbols

\usepackage{amsfonts}
\usepackage{todonotes}
\usepackage{hyperref}
\usepackage{cleveref}  % For referencing equations
\usepackage{cite} % For bibtex

% For writing todo inline
\presetkeys%
    {todonotes}%
    {inline}{}

% Used for writing pseudo code
\usepackage{algorithm}
\usepackage{algpseudocode}
% defined for writing input/output in pseudo code
\algblock{Input}{EndInput}
\algnotext{EndInput}
\algblock{Output}{EndOutput}
\algnotext{EndOutput}
\newcommand{\Desc}[2]{\State \makebox[6em][l]{#1}#2}

\usepackage[smartEllipses]{markdown}

\renewcommand\qedsymbol{$\blacksquare$}
\renewenvironment{proof}{{\textit{Proof} \\}}{\qed}

\newtheorem{definition}{Definition}[section]
\newtheorem{theorem}{Theorem}

\newtheorem{corollary}{Corollary}[section]
\newtheorem{lemma}{Lemma}[section]


\title{Differentially private vector aggreation in the case of multivariate gaussian data}
\author{Tim Sehested Poulsen}

\begin{document}

\maketitle

\section{Preliminaries}
\subsection{Definitions}

\textbf{The dataset} is denoted by $X$, where 
$X \in \R^{n \times d}$.
I have that $n$ denotes the number of entries in the dataset and 
$d$ is the number of dimensions of the dataset.
I will throughout the report refer to a single entry of 
the dataset as $x_i$ and a single dimension of the dataset as $\xj$, 
and therefore $x^{(j)}_i$
denotes the $j$'th dimension of the $i$'th entry.
\vspace*{0.3cm}

\textbf{Differential privacy} is the heuristic of 
releasing a database statistic whilst limiting the impact
of any one entry. It builds on the intuition that computing
a statistics on a private dataset should not reveal 
any sensitive information about any one individual 
as long as that individual has little to no effect on the outcome.
Differential privacy has multiple slightly different
formal definitions, 
one such is $(\epsilon, \delta)$-Differential Privacy
refered to as \edp which will be introduced later on.
A prerequisite for almost all of the different differential privacy
definitions relies on the concept of neighbouring dataset.
\vspace*{0.3cm}

\begin{definition}[Neighbouring dataset \cite{dwork2016}]
Two dataset $X, X' \in \R^{n \times d}$ are said to be 
neigbouring if they differ in at most a single entry.
Neighbouring dataset are denoted with the relation $X \sim X'$ and defined as followed
\[ X \sim X' \iff \lra{ \setdef{i \in \N}{i \le n \land x_i \ne x_i' } } \le 1 \]
\end{definition}

\begin{definition}[Sensitivity \cite{dpbasic}] % Definition 3.8
Let $f(X): \R^{n \times d} \rightarrow \R^k$ be a function. 
The $l_p$-sensitivity of $f$ is the maximal 
possible $l_p$-norm of the difference between the output of $f$ 
on two neighbouring dataset.
We denote the sensitivity as 
\[
\Delta_p (f) = \max_{X \sim X'} \| f(X) - f(X') \|_p 
\]
and then the total $l_2$-sensitivity is then
\end{definition}
Throughout the report I will only be working with $l_2$-sensitivity and
will just denote this as $\Delta(f)$ for ease of notation.
    

\vspace*{0.3cm}
\begin{definition}[$(\epsilon, \delta)$-Differential Privacy \cite{dwork2016}]

A randomized algorithm $\M: \R^{n \times d} \rightarrow \mathcal{R}$ 
is $(\epsilon, \delta)$-differentially private if for all possible 
subsets of outputs $S \subseteq \mathcal{R}$ and all pairs of 
neighbouring dataset $X \sim X'$ we have that
\[ \prob{M(X) \in S} \le e^{\epsilon} \cdot \prob{M(X') \in S} + \delta \]

\end{definition}

\begin{theorem}[ \edp under post-processing \cite{dpbasic}] %Proposition 2.1 in the cite
\label{theo:PostProc}

Let $\M: \R^{n \times d} \rightarrow \mathcal{R}$ be an \edp 
algorithm. Let $f: \mathcal{R} \rightarrow \mathcal{R}'$ 
be an arbitrary mapping, then 
$f \circ \mathcal{M}: \R^{n \times d} \rightarrow \mathcal{R}'$ is \edp.

\end{theorem}

\begin{proof}

Fix any pair of neighbouring datasets $X \sim X'$ and 
let $S \subseteq \mathcal{R}'$ be an arbitrary event. We then define 
$T = \setdef{r \in \mathcal{R}}{f(r) \in S}$.
We thus have that
\begin{align*}
    &\prob{f(\mathcal{M}(X)) \in S} = \prob{\mathcal{M}(X) \in T} \\
    &\le e^{\epsilon} \cdot \prob{\mathcal{M}(X') \in T} + \delta = 
    e^{\epsilon} \cdot \prob{f(\mathcal{M}(X')) \in S} + \delta
\end{align*}
    
\end{proof}

    

\paragraph{Error Measure}
As this report concerns itself exclusively with the sum of entries 
in a dataset, error will be defined as the expected
squared $l_2$-norm between the true sum and 
the output of a randomized algorithm.
So let $X \in \R^{n \times d}$ be the dataset and
$f(X) = \sum_i^n x_i$ be the true sum of all entries. The error of a
randomized algorithm $M : \R^{n \times d} \rightarrow \R^d$
which estimates $f(X)$ is then 
\[
    \text{Err}(M) := \ee{M(X) - f(X)}
\]

\paragraph{extra}

\subsection{Quadratic forms of random variables}
\todo{write cohesive text here, and decide what to keep}

\begin{lemma}
\label{lem:GaussTrans}
Let $X \sim \mathcal{N}(\mu, \sigma^2)$ be a gaussian random variable and let $r \in \R$ be a constant.
We then have that
\begin{align*}
    &rX \sim \mathcal{N}(\mu, (r\sigma)^2) \\
    &X - r \sim \mathcal{N}(\mu-r, \sigma^2) 
\end{align*}
    
\end{lemma}

Quadratics of random variables have been well studied \cite{BatesQuadForm,MathaiQaudForms}, 
specially in the case of multivariate gaussian varaibles \cite{IowaQuadNormForms,MathaiQaudForms}.
Even more research has been done in evaluating the CDF of these quadratic forms
for Gaussian random vectors \cite{QuadFormsNume,QaudFormsBounds}.

\begin{theorem}[Expectation of a quadratic random variable \cite{BatesQuadForm}]
\label{theo:ExpQuad}
Let $X$ be a $d$-dimensional random vector with expected value $\Exp{X} =  \bm{\mu_{X}}$
and covariance matrix $\var{X} = \bm{\Sigma_{X}}$. Let also $A$ be a constant 
$d \times d$ symmetric matrix, then 
\[
    \Exp{X^T A X} = tr \lr{A \bm{\Sigma_X}} + \bm{\mu}^T A\bm{\mu}
\]
\end{theorem}

\begin{proof}
Blah blah
\begin{align*}
    &\Exp{X^T A X} 
    =\text{tr}\lr{\Exp{X^T A X}} 
    =\Exp{\text{tr} \lr{X^T A X}} \\
    &= \Exp{\text{tr} \lr{A X X^T}}
    = \text{tr} \lr{A \Exp{X X^T}} 
    = \text{tr} \lr{A \lr{ \var{X} + \bm{\mu} \bm{\mu}^T}} \\
    &= \text{tr} \lr{A \bm{\Sigma}} + \text{tr} \lr{A \bm{\mu \mu^T}}
    = \text{tr} \lr{A \bm{\Sigma}} + \bm{\mu}^T A \bm{\mu}
\end{align*}
blah
\end{proof}
\begin{corollary}
\label{cor:expNorm}
Let $X \sim \mathcal{N}(\bm{0}, \bm{\Sigma_X})$ be a $d$-dimensional gaussian vector
with expected value $\bm{0}$, and let $\sigma_j^2$ denote the variance of the 
$j$'th dimension where $1 \le j \le d$.
By theorem \ref{theo:ExpQuad} we have that the expected $l_2$-norm 
of such a vector is given by
\[
    \ee{X} = \text{tr} (\bm{\Sigma_X}) = \sum_{j=1}^d \sigma_j^2
\]
\end{corollary}


\section{Algorithms}
\subsection{The Gaussian Mechanism}
One of the most foundational algorithms for achieving 
\edp is the Gaussian Mechanism \cite{dpbasic}. It computes the real
value of a statistic, where the $l_2$-sensitivity is known.
That is it produces a \edp estimate of a function 
$g: \R^{n \times d} \rightarrow \R^{d}$ where the $l_2$-sensitivity $ \Delta(g) $ 
is known.
It does so by computing the value of $g(X)$ and then adding noise
to each dimension drawn from the normal distribution 
$\mathcal{N}(0, \sigma_{\epsilon,\delta}^2)$.
This can be seen as adding a noise vector $\eta$ which 
is then distributed according to
the multivariate normal distribution 
$\mathcal{N}(\bm{0}, \sigma_{\epsilon,\delta}^2I)$. 
Pseudo code for the algorithm can be seen in Algorithm \ref{alg:gaussmech}.

\begin{algorithm}
\caption{The Gaussian Mechanism}\label{alg:gaussmech}
\begin{algorithmic}
    \Input
    \Desc{$\sigma_{\epsilon,\delta}$}{Standard deviation required to achieve \edp}
    \Desc{$X \in \R^{n \times d}$}{Dataset}
    \EndInput
    \Output
    \State\edp estimate of $g(X)$
    \EndOutput
    \State$\eta \gets \textnormal{sample from } \mathcal{N}(\overrightarrow{0}, \sigma_{\epsilon,\delta}^2I)$ \\
    \Return$g(X) + \eta$
\end{algorithmic}
\end{algorithm}
\noindent The algorithm quite intuitively prodcues error which is purely given by the norm of the noise added, and the expected error can be calculated to be
\[
    \ee{(g(X) + \eta) - g(X)} = 
    \ee{\eta}
\]
Which by corollary \ref{cor:expNorm} is 
\begin{equation}
\label{eq:ExpErrGM}
    \ee{\eta} = \sum_i^d \sigma_{\epsilon,\delta}^2 = d \cdot \sigma_{\epsilon,\delta}^2
\end{equation}
It is apparent that the main difficulty of the mechanism 
lies in determining a $\sigma_{\epsilon, \delta}$ which achieves
\edp, and preferably the smallest such one.

The following theorem was initially proven
\begin{theorem}\textnormal{\cite{dpbasic}}
\label{theo:gaussMech}
Let $g: \R^{n \times d} \rightarrow \R^{d}$ be an arbitrary 
$d$-dimensional function with $l_2$-sensitvity 
$ \Delta(g) = \max_{X \sim X'} \| g(X) - g(X') \|$, 
and let $\epsilon \in (0,1)$.
The Gaussian Mechanism with
$\sigma_{\epsilon, \delta} = \Delta(g)  \sqrt{2\ln(1.25/\delta)}/\epsilon$ 
is \edp.
\end{theorem}
The proof is rather long and is therefore ommitted here. A few years later it was shown in \cite{BalleWang} how to compute the minimal $\sigma_{\epsilon, \delta}$.
\begin{theorem}\textnormal{\cite{BalleWang}}
\label{theo:OptSig}
The Gaussian Mechanism is differentially private if and only if $\sigma_{\epsilon, \delta} \ge \Delta(g) \cdot \sigma_{opt}$ where
$\sigma_{opt} \in \R$ is the smallest value greater than 0, which satisfies
\[
    \Phi \lr{\frac{1}{2\sigma_{opt}} - \epsilon\sigma_{opt}} - e^{\epsilon} \Phi \lr{-\frac{1}{2\sigma_{opt}} - \epsilon\sigma_{opt}} \le \delta
\]
\end{theorem}
In \cite{BalleWang} it is also shown how to compute this value, and since this is of no importance to this project it will therefore not be covered.
For the rest of the report I will only be reffering to $\sigma_{\epsilon, \delta}$ as the minimal value given by theorem \ref{theo:OptSig} 
and not that given by theorem \ref{theo:gaussMech}.


The main downside of the Gaussian Mechanism is that it adds equal noise to all dimensions, regardless of the sensitivity in that dimension.
As this report focuses on giving differentially private estimates of sums of vectors, it is natural to ask whether adding equal noise in all dimensions
is optimal in this setting. Let us define the function of interest $f(X) = \sum_{i \in [n]} x_i$, for a dataset $X \in \R^{n \times d}$. 
The sensitivity of this function must be given by the largest possible norm of any vector.
Therefore if the largest difference between any two neighbouring datasets 
in the $j$'th dimension is given by 
\begin{equation}
\label{eq:DimCon}
    \Delta_j := \max_{X \sim X'}|X^{(j)} - X'^{(j)}|
\end{equation}
and we say that $X$ and $X'$ differ in the $i$'th entry we conclude that
\begin{equation}
\label{eq:DelF}
    \Delta(f) = \max_{X \sim X'} \| f(X) - f(X') \| = 
    \max_{X \sim X'} \| x_i - x_i' \| = \sqrt{\sum_{j \in [d]} \Delta_j^2} = \|\bm{\Delta}\|
\end{equation}
This means that by equation \ref{eq:ExpErrGM} the expected error of the Gaussian Mechanism when estimating $f(X)$ is given by
\begin{equation}
\label{eq:GMErr}
    \ee{\eta} = d \cdot \sigma_{\epsilon, \delta}^2 = d \cdot \lr{\Delta(f) \sigma_{opt}}^2 =
    d  \cdot \sigma_{opt}^2 \cdot \| \bm{\Delta} \|^2 
\end{equation}
A logical next step would be to add noise to each dimensions such that it is proportional to the sensitivity of that dimension.
This has been studied in \cite{Lebeda2022} and lays the foundation for this report. Their mechanism, appropriately called the Elliptical Gaussian Mechanism
works very similairly to the Gaussian Mechanism as described by algorithm \ref{alg:gaussmech}, though instead of sampling all $\eta_j$ from 
$\mathcal{N}(0,\sigma_{\epsilon, \delta}^2)$, here they are instead drawn from $\mathcal{N}\lr{0,\lr{\sigma_{opt}\cdot \frac{\Delta_j}{b_j}}^2}$. 
The algorithm is described in detail in Algorithm \ref{alg:EllipGM}

\begin{algorithm}
\caption{The Elliptical Gaussian Mechanism}\label{alg:EllipGM}
\begin{algorithmic}
    \Input
    \Desc{$\sigma_{opt}$}{Standard deviation as defined by theorem \ref{theo:OptSig}}
    \Desc{$X \in \R^{n \times d}$}{Dataset}
    \Desc{$\bm{b} \in \R^{d}$}{Scaling vector, where $\|\bm{b}\| = 1$}
    \Desc{$\bm{\Delta} \in \R^{d}$}{Sensitivies of all dimensions}
    \EndInput
    \Output
    \State \edp estimate of $f(X)$
    \EndOutput
    \For{$j \in [d]$}
        \State $\sigma_j \gets \sigma_{opt} \cdot \frac{\Delta_j}{b_j}$
        \State $\eta_j \gets \textnormal{sample from } \mathcal{N}(0, \sigma_j^2)$
    \EndFor \\
    \Return $g(X) + \eta$
\end{algorithmic}
\end{algorithm}

\todo{add lemma/theorem/proof of why this is \edp (figure out if needed). It is very similair to why it works for Gaussian data.}

The main thing that really differentiates itself from the normal Gaussian Mechanism is the introduction of the scaling vector $\bm{b}$.
This is the scaling of how much weight should be attributed to each dimension when adding noise. It is shown how to determine the optimal values for $b_j$
in \cite{Lebeda2022}, and what the expected error of the mechanism then is.

\begin{theorem}[Optimality and error of the Elliptical Gaussian Mechanism \cite{Lebeda2022}]
The value for $b_j$ which minimizes the expected $l_2$ error $\ee{\eta}$ of the Elliptical Gaussian Mechanism is as follows
\[
    b_j = \sqrt{\frac{\Delta_j}{\sum_{j \in [d]} \Delta_j}}
\]
which leads the error to be
\begin{equation}
\label{eq:EGMErr}
    \ee{\eta} =  \sigma_{opt}^2 \cdot \| \bm{\Delta} \|_1^2
\end{equation}
where $\| \cdot \|_1$ is the $l_1$ norm.

\end{theorem}
\todo{write the proof or skip it, it is going to be very similair to that which I will make}

Comparing the expected error between Algorithm \ref{alg:gaussmech} and Algorithm \ref{alg:EllipGM} we have the following ratio
\[
    \frac{d \cdot \sigma_{opt}^2 \cdot \| \bm{\Delta} \|^2 }{\sigma_{opt}^2 \cdot \| \bm{\Delta} \|_1^2} = \lr{\frac{\sqrt{d}\| \bm{\Delta} \|}{\| \bm{\Delta} \|_1}}^2
\]
in which it can be seen that they are equal when all entries of $\bm{\Delta}$ are the same. Otherwise the error for the Elliptical Gaussian Mechanism is lower
when $\bm{\Delta}$ is skewed. As argued in \cite{Lebeda2022} Algorithm \ref{alg:EllipGM} improves Algorithm \ref{alg:gaussmech} by a factor in $[1,d)$.

\section{Problem setup}
As previously mentioned the problem investigated here consists of realeasing the sum of vectors 
in a dataset under differential privacy.
More formally we whish to release the value of 
$ f(X) = \sum_{i = 1}^n x_i  $
under \edp. \\
The common factor for achieving \edp in both the Gaussian Mechanism
and the Elliptical Gaussian Mechanism is the knowledge that data lie within some bounds.
Specifically for the Elliptical Gaussian Mechanism data is required to lie within some hyperrectangle.
It is formally described by equation \eqref{eq:DimCon} 
essentially saying that there is an upper and lower bound on each dimension.
This requirement is needed to know the $l_2$-sensitivity $ \Delta(f) $ as shown in equation \eqref{eq:DelF}.
In this project I will change this assumption and instead 
look at the case where each dimension is normally distributed.
This means that for each 
$ j \in [d] $ we have that 
$x_{i,j} \sim \mathcal{N}(\mu_j, \sigma_j^2)$.
An equivalent formulation is that the data is 
multivariately gaussianly distributed but with no correlation between dimensions.
This means that $x_i \sim \mathcal{N}(\bm{\mu}, \Sigma ) $, 
where $\Sigma$ is a diagonal matrix with the variance of each 
dimension along its diagonal.
It is quite apparent that determining a limit $ \Delta_j $ is impossible 
in this setting as the Gaussian distribution is continously defined 
on the range $ (-\infty, \infty)$.
Several recent papers has combatted this by doing something called 
\textit{clipping} \cite{Huang2021,coinpress}. 
Clipping is the process of limiting the norm of any one entry 
to be at most a chosen threshold $C$. This means that every vector is transformed as such
\[
    \widehat{x}_i := \min \lrc{\frac{C}{\| x_i \|}, 1} \cdot x_i
\]
Clipping entries by a factor $C$ thus means that $ \Delta(f) = 2C $ 
as any one entry cannot have more impact on the summation than $C$.
If the summation $f(X)$ is 
performed on a clipped dataset $\widehat{X}$ it is equivalent to
defining the summation function $\widehat{f}: \R^{n \times d} \rightarrow \R^d$ as
\[
    \widehat{f}(X) = \sum_i^n \min \lrc{ \frac{C}{\| x_i \|}, 1} \cdot x_i
\]
Then by theorem \ref{theo:OptSig} the 
gaussian mechanism with the function $\widehat{f}$ is \edp with
$\sigma_{\epsilon, \delta} = \Delta\lr{\widehat{f}} \sigma_{opt} = 2C \sigma_{opt}$.
Though the mechanism is still \edp it will now have 
a larger error when regarding the true sum
$f(X) = \sum_i^n x_i$ as the actual answer. If the probability of clipping is set
so low that we actually don't expect to clip any entries we can use $\widehat{f}$ as an approximation of $f$.
Say there are $n$ points in a dataset, I will thus set the probability of clipping to be less than $\frac{1}{n}$ and get that
\[
    \ee{\lr{\widehat{f}(X) + \eta} - f(X)} \approx \ee{\lr{f(X) + \eta} - f(X)} = \ee{\eta}
\]
\todo{rewrite and introduce better}
In this setting, we again have the intuition that addding equal noise in all dimensions is non optimal, and instead the noise in a dimension should be proportional
to the variance $\sigma_j^2$ in that dimension. In a desire to achieve similair results to that of the Elliptical Gaussian Mechanism, just with normally distributed data,
I will use somewhat the same approach to achieve \edp.
They achieve \edp by finding a transformation of points such that they all lie within the unit ball centered at the origin. 
There does not exist such a transformation which is linear (I think I mean Continous/Homeomorphic), as there will always be a non-zero 
probability of observing points outside the unit ball.
Instead I will introduce the constraint that the expected norm of vectors after the transformation should be $1$. 
I whish to find a scaling of each dimension $b_j$ s.t.
\begin{align}
\label{eq:Alg3Con}
    &\Exp{\| x_i \odot \bm{b} \|} = 1  \iff \ee{x_i \odot \bm{b}} = \sum_{j \in [d]} (\sigma_jb_j)^2 = 1
\end{align}
where $\bm{b} = (b_1, b_2, \dots, b_d)$, $\odot$ is the element-wise product, and $\sigma_j$ is the standard deviation of the $j$'th dimension.
Then $\widehat{f}$ can be computed on this transformed dataset, where the probability of clipping is less than $\frac{1}{n}$.
As Theorem \ref{theo:PostProc} shows, \edp is preserved under post processing, 
we can therefore add noise to each coordinate drawn from $\mathcal{N}(0,(2C\sigma_{opt})^2)$ in the transformed space to achieve \edp.
The transformation back to the original space is then done by multiplying each dimension with $b_j^{-1}$, and due to the linearity of transformation 
this is also done to the noise added.
We end up with the following noise vector being added
\[
    \bm{\eta} = \lr{ b_1^{-1}\eta_1, b_2^{-1}\eta_2, \dots, b_d^{-1}\eta_d }
\]
in which all $\eta_j \sim \mathcal{N}(0,(2C\sigma_{opt})^2)$, and then by lemma \ref{lem:GaussTrans} we have $b_j^{-1}\eta_j \sim \mathcal{N}(0, (b_j^{-1} \cdot 2C\sigma_{opt})^2)$.
As the error is approximately given by $\| \bm{\eta} \|$ we have due to Corollary \ref{cor:expNorm} that the expected error is 
\begin{equation}
\label{eq:Alg3Err}
    \ee{\bm{\eta}} = \sum_{j=1}^d \lr{b_j^{-1} \cdot 2C\sigma_{opt}}^2   = \lr{2C\sigma_{opt}}^2 \sum_{j=1}^d b_j^{-2}
\end{equation}
\todo{Give pseudo code for algorithm}
Now we are back to a very similiar problem to the one solved in \cite{Lebeda2022}. Minimize the error described in equation \eqref{eq:Alg3Err}
under the constraint given in equation \eqref{eq:Alg3Con}. This leads to the following Lemma

\begin{lemma}
\label{lem:Optb}
Let $X \in \R^{n \times d}$ be a dataset where each dimension is independently gaussianly distributed, i.e.
$x_{i,j} \sim \mathcal{N}(\mu_j, \sigma_j^2)$. Then the expected error of Algorithm \ref{alg:EGM2} is minimized when
\[
    b_j = \frac{1}{\sqrt{\sigma_j} \sqrt{\sum_{i=1}^d \sigma_i}}
\]
\end{lemma}
\begin{proof}
Using lagrangian multipliers we find the local maxima or 
minia of the function subject to equality constraints.
To do so we construct the lagrangian function $\mathcal{L}: \R^{d+1} \rightarrow \R$ from the optimization problem given by
equation \eqref{eq:Alg3Err} and the constraint given by \eqref{eq:Alg3Con} for ease of notation we define 
$\sigma_{\epsilon, \delta} := 2C\sigma_{opt}$.
\[
    \mathcal{L}(\bm{b},\lambda) = \sigma_{\epsilon,\delta}^2\sum_{j \in [d]} b_j^{-2}
    + \lambda \lr{ \sum_{j=1}^d \lr{\sigma_j b_j}^2 - 1}
\]
We then find the stationary points of it,
by setting the derivative of it to $\bm{0}$.
The derivative with respect to $b_j$ is 
\[
    \frac{\partial \mathcal{L}}{\partial b_j} (\bm{b}, \lambda)= 
    \frac{\partial}{\partial b_j} \lr{\sigma_{\epsilon,\delta}^2 \cdot b_j^{-2}
    + \lambda \lr{\sigma_j b_j}^2} =
    -2 \sigma_{\epsilon,\delta}^2 b_j^{-3} + 2 \lambda \sigma_j^2b_j
\]
I then solve $\frac{\partial \mathcal{L}}{\partial b_j} = 0$ for $b_j$
\begin{align}
    &-2\sigma_{\epsilon,\delta}^2 b_j^{-3} + 2 \lambda \sigma_j^2b_j = 0 \iff
    \lambda \sigma_j^2b_j = \sigma_{\epsilon,\delta}^2 b_i^{-3}  \\
\label{eq:solvbi}
    &\iff b_j^4 = \frac{\sigma_{\epsilon,\delta}^2}{\lambda \sigma_j^2} \iff 
    b_j = \frac{\sqrt{\sigma_{\epsilon,\delta}}}{\lambda^{\frac{1}{4}} \sqrt{\sigma_j}} 
\end{align}
I now have the last partial derivative 
$\frac{\partial \mathcal{L}}{\partial \lambda} = 0$ which I solve for
$\lambda$ using the previous expression for $b_j$.

\begin{align*}
    &\frac{\partial \mathcal{L}}{\partial \lambda} = 
    \sum_{j=1}^d \lr{ \sigma_j b_j}^2 - 1 \\
    &\sum_{j=1}^d \lr{ \sigma_j b_j}^2 - 1 = 0 \iff
    \sum_{j=1}^d \sigma_j^2 \lr{\frac{\sigma_{\epsilon,\delta}}{\sqrt{\lambda}\sigma_j}} = 1 \iff \\
    &\frac{\sigma_{\epsilon,\delta}}{\sqrt{\lambda}} \sum_{j=1}^d \frac{\sigma_j^2}{\sigma_j} = 1 \iff
    \sigma_{\epsilon,\delta} \sum_{j=1}^d \sigma_j = \sqrt{\lambda}
\end{align*}
Inserting back into equation \ref{eq:solvbi}
\[
    b_j = \frac{\sqrt{\sigma_{\epsilon,\delta}}}{\lambda^{\frac{1}{4}} \sqrt{\sigma_j}} =
    \frac{\sqrt{\sigma_{\epsilon,\delta}}}{\sqrt{\sigma_{\epsilon,\delta} \sum_{i=1}^d \sigma_i} \sqrt{\sigma_j}} = 
    \frac{1}{\sqrt{\sigma_j} \sqrt{\sum_{i=1}^d \sigma_i} } 
\]
\todo{show that this stationary point is a minimum}
Giving us the value of $b_j$ which minimizes the expected error.
\end{proof}

Now to evaluate the expected error of the mechanism, which is given by equation \eqref{eq:Alg3Err}, it is highly important important to determine the value $C$.
As $C$ should be given by the minimal value which satisfies the inequality
\[
   \prob{ \| x_i \odot \bm{b} \| > C } = \prob{ \| x_i \odot \bm{b} \|^2 > C^2} < \frac{1}{n}
\]
again where $n$ denotes the number of points in the dataset.
As $x_i$ is a multivariate gaussian distribution, so is $x_i \odot \bm{b}$, and $\|x_i \odot \bm{b}\|^2$ is the sum of independent gaussian variables squared.
Such a sum is distributed as a generalized Chi-square \cite{GenChiSq}, and neither its PDF nor CDF has a closed form.
These quadratic forms of gaussian vectors has been studied well and for decades \cite{MathaiQaudForms}, and some special cases does have closed forms.
Some have also studied giving tail bounds on these probabilities \cite{QaudFormsBounds}, unfortunately there was none which were of use in this setting.
However, there does exist several numerial algorithms for evaluating the cumulative density function with high precision \cite{QuadFormsNume}. 
I would recommend using one of these algorithms in practice, but for the sake of analysis I will provide and upper bound on $C$ using Bernstein's inequality.
\todo{Should I give the general definition of Bernsteins inequality before here? Or just inside the proof}

\begin{lemma}
\label{lem:Bernstein}
Let $X_1, X_2, \dots , X_d$ be $d$ independent random gaussian variables 
where for $ 1 \le j \le d$ we have that $X_j \sim \mathcal{N}(0, \sigma_j^2)$.
Let thereafter the largest standard deviation be denoted as $\sigma_* := \max_{j \in [d]} \sigma_j$,
we can then bound the probability for the sum of variables squared as follows 
\[
\prob{\sum_{j \in [d]} X_j^2  \ge 
    t \sqrt{8 \sum_{j \in [d]} \sigma_j^4} +
    \sum_{j \in [d]} \sigma_j^2 } < e^{-t^2}
\]
for
\[
0 \le t \le 
    \frac{1}{6 \sigma_*^2} \sqrt{2 \sum_{j \in [d]} \sigma_j^4}
\]
\end{lemma}
\begin{proof}
At first we define the random variable $Y_j = X_j^2 - \Exp{X_j^2}$ 
using the $j$'th gaussian random variable.
As $\Exp{Y_j} = \Exp{X_j^2 - \Exp{X_j^2}} = \Exp{X_j^2} - \Exp{X_j^2} = 0$ 
we have that $Y_j$ is zero centered. We are thus interested in giving bounds on
$\prob{\sum_{j \in [d]} Y_j}$. 
We can use Bernsteins inequality, if the following constraint holds for all $k \in \N$ with $k \ge 2$
 and for all $j \in [d]$, and for some $L \in \R$
\begin{equation}
\label{eq:BernCon}
    \Exp{ |Y^k_j| } \le \frac{1}{2} \Exp{Y_j^2} L^{k-2} k!
\end{equation}
Initially we have that $\Exp{ |Y_j^k| } = \Exp{ |\lr{X_j^2 - \Exp{X_j^2}}^k|}$
and since $X_j^2 \ge 0$ and therefore also $\Exp{X_j^2} \ge 0$ 
we can therefore bound it by

\[
    \Exp{ |\lr{X_j^2 - \Exp{X_j^2}}^k|} \le \Exp{ | X_j^{2k} | } =
    \Exp{ | \lr{\sigma_j^2 Z}^{k} | } = 
    \sigma_j^{2k} \cdot \Exp{ |Z^{k} | }
\]
Where $Z \sim \chi^2_1$ is a chi square with 1 degree of freedom.
The moment generating function of $Z$ is 
$\Exp{Z^m} = 1 \cdot 3 \cdot 5 \cdot \ldots \cdot (2m - 1) $.
Using this we can get the following bound
\begin{align*}
    &\sigma_j^{2k} \cdot \Exp{ |Z^{k} | } =
    \sigma_j^{2k} \cdot \prod_{c=1}^k(2c-1) =
    \sigma_j^{2k} \cdot 3\cdot \prod_{c=3}^k (2c-1) \\
    &\le \sigma_j^{2k} \cdot 3 \cdot \prod_{c=3}^k 2c
    = \frac{3}{2}\sigma_j^{2k} \cdot 2^{k-2} \cdot k!
\end{align*}
Concluding that $\Exp{ |Y_j^k| } \le \frac{3}{2}\sigma_j^{2k} \cdot 2^{k-2} \cdot k!$.

\noindent Secondly I calculate $\Exp{Y_j^2}$ exactly to be used in equation \ref{eq:BernCon}
\begin{align}
\label{eq:ExpY2:1}
    &\Exp{Y_j^2} = 
    \Exp{\lr{X_j^2 - \Exp{X_j^2}}^2} \\
    &=\Exp{X_j^4 + \Exp{X_j^2}^2 - 2X_j^2\Exp{X_j^2}} \\
    &=\Exp{X_j^4} + \Exp{X_j^2}^2 - 2\Exp{X_j^2}^2 \\
    &=\Exp{X_j^4}  - \Exp{X_j^2}^2 =
    \sigma_j^4\Exp{Z^2}  - \lr{\sigma_j^2\Exp{Z}}^2 \\
\label{eq:ExpY2:2}
    &=3\sigma_j^4  - \sigma_j^4 =
    2\sigma_j^4
\end{align}

Equation \eqref{eq:BernCon} is therefore rewritten to 
\begin{align*}
    \frac{3}{2}\sigma_j^{2k} \cdot 2^{k-2} \cdot k! \le \frac{1}{2} \cdot 2 \sigma_j^4 L^{k-2} k! 
    \iff \frac{3}{2} \sigma_j^{2k-4} \cdot 2^{k-2} \le L^{k-2}
\end{align*}
This expression is then split into two cases, when $k=2$, in which it can be seen from equation \eqref{eq:BernCon} that this holds for any $L$.
The other case is $k>2$, where we have that 
\[
    \lr{\frac{3}{2}}^{\frac{1}{k-2}} \sigma_j^2 \cdot 2 \le L
\]

As $\lim_{k \rightarrow \infty} \lr{\frac{3}{2}}^{\frac{1}{k-2}} = 1$, 
and these constraints must hold for all $j \in [d]$, we can define $\sigma_* = \max_{j \in [d]} \sigma_j$ and finally have that
\begin{align*}
    &L = 3\sigma_*^2 
\end{align*}

We can then use Bernstein's inequality to bound the following 
\begin{equation}
\label{eq:BernIneq}
    \prob{\sum_{j \in [d]} Y_j \ge 2 t \sqrt{ \sum_{j \in [d]} \Exp{Y_j^2}}} < e^{-t^2}
\end{equation}

Which in in this case can be rewritten to get a tail bound on $\sum_{j \in [d]} X_j^2$,
by reusing results from equations \eqref{eq:ExpY2:1}-\eqref{eq:ExpY2:2}
\begin{align*}
&\prob{\sum_{j \in [d]} Y_j \ge 2 t \sqrt{ \sum_{j \in [d]} \Exp{Y_j^2}}} 
=\prob{\sum_{j \in [d]} \lr{X_j^2 - \Exp{X_j^2}} \ge 2 t \sqrt{ \sum_{j \in [d]} 2 \sigma_j^4}} \\
&=\prob{\sum_{j \in [d]} X_j^2  \ge 2 t \sqrt{ \sum_{j \in [d]} 2 \sigma_j^4} + \sum_{j \in [d]} \Exp{X_j^2}} 
=\prob{\sum_{j \in [d]} X_j^2  \ge 2 t \sqrt{ \sum_{j \in [d]} 2 \sigma_j^4} + \sum_{j \in [d]} \sigma_j^2}  \\
&=\prob{\sum_{j \in [d]} X_j^2  \ge 
    t \sqrt{ 8 \sum_{j \in [d]}  \sigma_j^4} +
    \sum_{j \in [d]} \sigma_j^2 } \\
\end{align*}

Finally giving us that 
\[
\prob{\sum_{j \in [d]} X_j^2  \ge 
    t \sqrt{ 8 \sum_{j \in [d]}  \sigma_j^4} +
    \sum_{j \in [d]} \sigma_j^2 } < e^{-t^2}
\]
as long as $t$ lies within the following bounds
\[
    0 \le t \le \frac{1}{2L} \sqrt{\sum_{j \in [d]} \Exp{Y_j^2}} =
    \frac{1}{6 \sigma_*^2} \sqrt{2 \sum_{j \in [d]} \sigma_j^4}
\]
\end{proof}

\noindent Combining lemma \ref{lem:Bernstein} with theorem \ref{lem:Optb} 
we have can conclude the following:
\todo{Introduce variables better}

\begin{theorem}
\label{theo:Alg3OptErr}
Let $X \in \R^{n \times d}$ be a dataset in which all dimensions are independently gaussian random variables, i.e. $x_{i,j} \sim \mathcal{N}(0, \sigma_j^2)$. 
Let $\sigma_* := \max_{j \in [d]} \sigma_j$ be the maximal standard deviation of all dimensions. We then have that
when $\ln (n) \le \frac{\sum_{j \in [d]} \sigma_j^2}{18 \sigma_*^2}$ the expected error of algorithm \ref{alg:EGM2} can be upper bounded by 
\[
    \ee{\bm{\eta}} \le 4\sigma_{opt}^2  \cdot \sum_{j \in [d]} \sigma_j \cdot
    \lr{ \sqrt{8\ln(n) \sum_{i \in [d]} \sigma_i^2} + \sum_{j \in [d]} \sigma_j }
\]
\end{theorem}

\begin{proof}
By equation \eqref{eq:Alg3Err} the error is given by
\begin{equation}
\label{eq:Alg3Err2}
    \ee{\bm{\eta}} = \lr{2C\sigma_{opt}}^2 \cdot 
    \sum_{i=1}^d b_j^{-2} 
\end{equation}
When clipping is performed to remove less than $n^{-1}$ we have the following 
\[
    \prob{ \| x_i \| \ge C} =
    \prob{ \| x_i \|^2 \ge C^2} =
    \prob{ \sum_{j \in [d]} x_{i,j}^2 \ge C^2}
\]
\todo{clean up sigma notation}
Which means I can give an upper bound on $C^2$ by using
lemma \ref{lem:Bernstein}, and inserting that $\sigma_j = \sigma_j \cdot b_j$ 
where $b_i$ is given by theorem \ref{lem:Optb}. I whish the clipping 
probability to be less than $n^{-1}$, which implies $t = \sqrt{\ln(n)}$. 
Combining these results I get that
\begin{align*}
    C^2 &\le \sqrt{8 \ln(n) \sum_{j \in [d]}  \lr{\sigma_j^4}} + 
    \sum_{j \in [d]} \sigma_j^2 \\
    &=\sqrt{8 \ln(n) \sum_{j \in [d]} \lr{\frac{\sigma_j}{\sum_{i \in [d]} \sigma_i}}^2} + 
    \sum_{j \in [d]} \frac{\sigma_j}{\sum_{i \in [d]} \sigma_i} \\
    &=\sqrt{8 \ln(n) \sum_{j \in [d]} \sigma_j^2} \cdot \frac{1}{\sum_{i \in [d]} \sigma_i}
    + 1
\end{align*}
Inserting this back into equation \ref{eq:Alg3Err2} we conclude
\begin{align*}
    \ee{\eta} &= 4\sigma_{opt}^2 \cdot 
    C^2 \cdot \sum_{i=1}^d b_j^{-2} \\
    &\le 4\sigma_{opt}^2 \cdot 
    \lr{\sqrt{8 \ln(n) \sum_{j \in [d]} \sigma_j^2} \cdot \frac{1}{\sum_{i \in [d]} \sigma_i}
    + 1} \cdot \sum_{i=1}^d \lr{\sigma_i \cdot \sum_{j \in [d]} \sigma_j} \\
    &\le 4\sigma_{opt}^2 \cdot 
    \lr{\sqrt{8 \ln(n) \sum_{j \in [d]} \sigma_j^2} \cdot \frac{1}{\sum_{i \in [d]} \sigma_i}
    + 1} \cdot \lr{\sum_{i=1}^d \sigma_i}^2 \\
    &= 4\sigma_{opt}^2 \cdot \sum_{i=1}^d \sigma_i \cdot 
    \lr{\sqrt{8 \ln(n) \sum_{j \in [d]} \sigma_j^2}
    + \sum_{i=1}^d \sigma_i}
\end{align*}
And the constraint on $t = \sqrt{\ln (n)}$ from lemma \ref{lem:Bernstein} is
\begin{align*}
    &0 \le t \le \frac{1}{6 \sigma_*^2} \sqrt{2\sum_{j \in [d]} \sigma_j^4} \iff \\
    &\sqrt{\ln (n)} \le \sqrt{\lr{\frac{\sum_{j \in [d]} \sigma_j}{\sigma_*}}^2 \cdot \frac{2}{36} \cdot \sum_{j \in [d]} \lr{\frac{\sigma_j}{\sum_{i \in [d]} \sigma_i}}^2} \iff \\
    &\ln (n) \le \frac{1}{18\sigma_*^2} \cdot \sum_{j \in [d]} \sigma_j^2 \iff \\
    &\ln (n) \le \frac{\sum_{j \in [d]} \sigma_j^2}{18 \sigma_*^2}
\end{align*}

At this point it would be preferable to compare the error of algorithm 3, to that of doing no transformation, and similairly deciding on a bound $C$ 
such that the clipping probability was less than $n^{-1}$. However as I can only give upper bounds on these errors there is little to no value in comparing 
upper bounds as there is no guarentee of the bounds. So even if I were to determine an upper bound for the error without transformation which was lower than that of algorithm 3, 
it could very well still be that the error was less in reality. Secondly to produce the upper bound in as in \ref{theo:Alg3OptErr} using Bernstein's inequality, I
have the constraint that all $\sigma_j \le \frac{2}{3}$, which is unlikely to happen in reality, whereas $\frac{\sigma_j}{\sum_{j \in [d]} \sigma_j} \le \frac{4}{9}$ 
is much more likely to be satisfied for all $j$.
As numerical solutions exist for evaluating $\prob{\|x_i\| > C^2}$ it is evident (oplagt) that emperically evaluating these errors and comparing them, would yield
whether doing the transformation is beneficial.
Intuitively it makes sense that allowing different noise to be added for each dimension, can only produce better results.
\end{proof}
\todo{Compare error with upper bound to the one which we be if no transformation was done.
 i.e clipping in the original space. Impossible due to the constraint on $\sigma_*$}
\todo{Write about alternative ways of giving upper bound on $C$, and why it is not so important}
\todo{Write alternative ways of defining the problem. CLip prob less than $n^{-1}$ and such}


\newpage
\subsection{Gaussian data}
Let $X^{(j)} \sim \mathcal{N}(0, \sigma_j^2)$ 
As the expected $l_2$-norm of $x_i$ is given by
\[
    \ee{x_i} = \sum_{j=1}^d \sigma_j^2
\]
To achieve an expected norm of $1$ I will scale 
each dimension by a factor $\frac{1}{b_j}$ which achieves this.
If 
\[
\hat{x_i} = \left( \frac{x_i^{(0)}}{b_0}, \frac{x_i^{(1)}}{b_1}, \dots, \frac{x_i^{(d)}}{b_d} \right)
\]
This means that $X^{(j)} \sim \mathcal{N}(0,\frac{\sigma_j^2}{b_j^2})$ 
and the expected norm is given by 
\[
    \ee{x_i} = \sum_{j=1}^d \frac{\sigma_j^2}{b_j^2}
\]
and I can introduce that constraint that the expected norm
after the transformation should be $1$.
In such a case when noise is added after the transformation
$\hat{X} + \eta$
where $\eta \sim N(0, t^2)$ and achieves \edp in this space
then then due to linearity of transformation the noise 
introduced in the original space is then given by
then the error is





I desire a transformation of $x_i^{(j)}$ such that the 
expected norm is $1$. Thus I must scale each dimension by
$\frac{1}{b_j}$, and have that 
\[
    \ee{x_i} = 1
\]

Minimize $\| \hat{\eta} \|$ under the constraint that $\ee{x_i} = 1$  

\begin{lemma}
\label{lem:chibound}
Let $X \sim \mathcal{N}(0,\sigma^2)$, and $\Phi$ denote the 
cumulative density function of $\mathcal{N}(0,1)$, then the 
cumulative density function of $X^2$ is given by
\[
    F_{X^2}(x) = \prob{X^2 \le x} = 2\Phi \lr{\frac{\sqrt{x}}{\sigma}} - 1
\]
\end{lemma}
\begin{proof}
\begin{align*}
    &\prob{X^2 \le x} =
    \prob{ |X| \le \sqrt{x}} =
    2 \prob{0 \le X \le \sqrt{x}}  \\
    & =2  \lr{\prob{X \le \sqrt{x}} - \prob{X \le 0}} =
    2 \lr{\prob{X \le \sqrt{x}} - \frac{1}{2}}  \\
    &= 2 \Phi \lr{\frac{\sqrt{x}}{\sigma}} - 1 
\end{align*}
\end{proof}
\begin{corollary}
From lemma \ref{lem:chibound} we can give following bound for $X \sim \mathcal{N}(0,\sigma^2)$.
\[
    \prob{X^2 > (4\sigma)^2} < 10^{-4}
\]
\end{corollary}


\newpage





Bernsteins inequality then states that
\[
    \prob{\sum_{j \in [d]} X_j \ge 2 t \sqrt{ \sum_{j \in [d]} \Exp{X_i^2}}} < e^{-t^2}
\]
for all 
\[
    0 \le t \le \frac{1}{2L} \sqrt{\sum}
\]

I have that $\bar{X}$ is a standard gaussian variable. \\
Let $Y_j = X_j^2$ I am interested in $\sum_{j \in [d]} Y_j$,
then it has 
\begin{align*}
    &\Exp{ |Y_j^k| } = \Exp{ |X_j^{2k}| } = \Exp{ |X_j|^{2k} } = 
    \Exp{ | \sigma_j \bar{X} |^{2k} } = \\
    &| \sigma_j |^{2k} \Exp{ | \bar{X} |^{2k}} = 
    \sigma_j ^{2k} \prod_{c=1}^k 2c = \sigma_j^{2k} \cdot 2^k \cdot k!
\end{align*}
Which means $\Exp{Y_j^2} = 8 \sigma_j^{4}$.
To check the constraint (find value for L)
\begin{align*}
    &\sigma_j^{2k} \cdot 2^k \cdot k! \le
    \frac{1}{2} \cdot 8 \sigma_j^{4} \cdot L^{k-2} k! \iff \\
    &\sigma_j^{2k} \cdot 2^k \le 4 \sigma_j^4 \cdot L^{k-2} \iff \\
    &\sigma_j^{2(k-2)} \cdot 2^k \le 4 \L^{k-2}
\end{align*}
We have that 
$\sigma_j = \sigma_j b_j = \frac{\sqrt{\sigma_j}}{\sqrt{\sum_{i \in [d]} \sigma_i}} = \sqrt{\frac{\sigma_j}{\sum_{i \in [d]} \sigma_i}}$.
Therefore
\begin{align*}
    &\sigma_j^{2(k-2)} \cdot 2^k  = 
    \lr{\sqrt{\frac{\sigma_j}{\sum_{i \in [d]}\sigma_i}}}^{2(k-2)} \cdot 2^k = \\
    &\lr{\frac{\sigma_j}{\sum_{i \in [d]}\sigma_i}}^{k-2} \cdot 2^k \le 2^k
\end{align*}
This therefore implies
\[
    2^k \le 4 L^{k-2} \iff
    2^{k-2} \le L^{k-2} \iff
    2 \le L
\]
If it is desired that less than $10^{-p}$ points are removed then 
$t = \sqrt{p \ln(10) }$ as long as 
\[
    \sqrt{p \ln(10)} \le \frac{1}{2L}\sqrt{\sum_{j \in [d]} \Exp{X_j^2}} 
    = \sqrt{4} \cdot \frac{\sqrt{\sum_{i \in [d]} \sigma_i^2}}{\sum_{i \in [d]} \sigma_i}
\] 



\paragraph*{An alternative}
Again minimize $\| \hat{\eta} \|$, but instead the constraint
comes from Chebyshev's inequality, I always have that
\[
    \prob{ |\|x_i\|^2 - \ee{x_i}| \ge k \cdot \sqrt{\var{\|x_i\|^2}}} \le \frac{1}{k^2}
\]
Therefore I can set $\frac{1}{k^2} = 0.05$, and find a 
transformation where I decide how many standard deviations
I must be away from the mean to have norm greater than 1, i.e.
\begin{align*}
    \ee{x_i} + k \cdot \sqrt{\var{\|x_i\|^2}} = 1 \implies k = \frac{1-\ee{x_i}}{\sqrt{\var{\|x_i\|^2}}}
\end{align*}
I then have that
\[
    \frac{1}{k^2} = \frac{1}{\lr{\frac{1-\ee{x_i}}{\sqrt{\var{\|x_i\|^2}}}}^2} = 
    \lr{\frac{\sqrt{\var{\|x_i\|^2}}}{1-\ee{x_i}}}^2 = 
    \frac{\var{\|x_i \|^2}}{\lr{1 - \ee{x_i}}^2}
\]
I already know that
\begin{alignat*}{2}
    &\ee{x_i} = &\sum_{j=1}^d \frac{\sigma_j^2}{b_j^2} \\
    &\var{\|x_i \|^2} = 2 &\sum_{j=1}^d \frac{\sigma_j^4}{b_j^4}
\end{alignat*}
I therefore have the constraint 
\[
    \frac{\var{\|x_i \|^2}}{\lr{1 - \ee{x_i}}^2} = 
    \frac{2 \sum_{j=1}^d \frac{\sigma_j^4}{b_j^4} \\}
    {\lr{1-\sum_{j=1}^d \frac{\sigma_j^2}{b_j^2} }^2} = 
    0.05
\]
Be aware that this could find cases where expected norm is greater than 1
and the constraint then says that they are never less than 1 in norm.
Another restriction could be that expected norm is < 1.

Chebyshev's inequality could also be used when expected norm should be 1
and then put a bound on number of std away one must be to have less than 
0.05 fraction of data removed.

\newpage


\newpage

\section*{Extra}
Determining $\alpha$ using Bernsteins inequality
\[
    \prob{\|x_i\|^2 - \ee{x_i} > t} < 2 \cdot \exp \lr{- \frac{t^2/2}{\var{\|x_i\|^2} + C \cdot t/3}}
\]
In the case where $\ee{x_i} = 1$ and $b_i$ is optimized in this case,
we have that
\begin{align*}
    &P(\|x_i\|^2 > 1 + t) < 2 \exp \lr{- \frac{t^2/2}{\var{\|x_i\|^2} + C \cdot t/3}} \\
    &\var{\| x_i \|^2} = 2 \frac{\sum_i^d \sigma_i^2}{\lr{\sum_i^d \sigma_i}^2} \\
    &C = \max_{i \in [d]} \lr{\sigma_i} \cdot \frac{16}{\sum_i^d \sigma_i}
\end{align*}
$C$ is decided such that less than $0.0001$ fraction of the data is outside this bound
in each dimension. i.e. 
\[
    \forall j \in [d] \text{ : }\prob{X^{(j)} > C} < 0.0001
\]

Solving for $t$
\begin{align*}
&2 \cdot \exp \lr{- \frac{t^2/2}{\var{\|x_i\|^2} + C \cdot t/3}} = 0.0001
\iff \ln(\frac{0.0001}{2}) = -\frac{t^2/2}{\var{\|x_i\|^2} + C \cdot t/3}
\implies \\ &t = - \frac{C \ln(\frac{0.0001}{2}) - 
\sqrt{\ln(\frac{0.0001}{2})(-18 \cdot \var{\|x_i\|^2} + C^2\ln(\frac{0.0001}{2}))}}{3}
\end{align*}
Then we have that $\alpha = 1 + t$.

\todo{Determine $\alpha$ for the bound using \url{https://en.wikipedia.org/wiki/Concentration_inequality}, or \url{https://web.stanford.edu/class/cs229t/2017/Lectures/concentration-slides.pdf}}


\bibliography{refs.bib}{}
\bibliographystyle{acm}


\end{document}

